\documentclass[../../Main/Main.tex]{subfiles}

\begin{document}

El desarrollo de un paquete computacional para la estimación del modelo resulto ser la tarea, al mismo tiempo, más retadora y cautivante de este trabajo. El código, conecta la teoría del modelo con el componente probabilista y lo lleva a un terreno práctico y más tangible. Durante este esfuerzo, se vio la necesidad de desarrollar funcionalidad adicional para ir probando cada parte del modelo, desde la estimación de la matriz $F$ hasta las proyecciones de ejemplos en 2D. La capacidad y flexibilidad del paquete se ven reflejadas en la simpleza de su uso en su estado actual. 

El paquete se desarrolló en el lenguaje de programación estadístico \verb|R| por dos razones: por la familiaridad con la que se contaba y por la practicidad del lenguaje para manejar objetos matemáticos como vectores, matrices y listas. 

%- Citar a H.Wickham y los paquetes que ayudaron a hacer el desarrollo 
%-Como descargarlo e instalar el paquete
%- Mencionar los métodos S3 de \verb|plot| y \verb|summary|, mencionar las clases
%- Mencionar que se trata de respetar la notación matemática del trabajo
%- Mencionar que en el paquete en si, hay una mejor descripción de los objetos y se hicieron usando el paquete Reoxygen 2



La idea del paquete es poder usar el modelo de una forma sencilla, asimismo, poderlo validar y explorarlo sin tener que programar demasiado. En general, se busca respetar la sintaxis clásica de \verb|R| para la estimación y exploración de objetos. En la Tabla \ref{tab:Codigo} se presenta un ejemplo \textit{mínimiamente funcional} de como se puede correr un \verb|bpwpm|.

\begin{table}[h]
\makebox[\linewidth]{\rule{\textwidth}{0.4pt}}
\begin{verbatim}
	mod <- bpwpm(datos y parámetros)
	summary(mod)
	plot(mod) 
	plot_2D(mod)	# Si los datos están en 2D 

	# Datos X y Y en muestra o fuera de ella	
	mod_res <- predict(mod,X,Y)
	summary(mod_res)
	plot(mod_res)
\end{verbatim}
\makebox[\linewidth]{\rule{\textwidth}{0.4pt}}
\caption{Ejemplo mínimamente funcional}
\label{tab:Codigo}
\end{table}


\subsection{Listado de funciones}
\subsubsection*{Función \textit{bpwpm\_gibbs}}
Encontrada en el archivo \verb|bpwpm_gibbs.R|.
\begin{itemize}[label={}]
	\item \verb|bpwpm_gibbs|($\ysn,X,M,J,K,\ldots$): esta es la función principal del paquete que realiza la simulación de la cadena de Markov usando los datos para el entrenamiento y los parámetros especificados.
\end{itemize}

\subsubsection*{Función \textit{predict.bpwpm}}
Encontrada en el archivo \verb|predict_funcs.R|.
\begin{itemize}[label={}]
	\item \verb|predict.bpwpm|(\verb|object|,$\,\tilde{y},\tilde{X\,}$,\verb|thin|, \verb|burn-in|, \verb|type|, $\ldots$): función genérica de la clase S3 que realiza predicciones de un nuevo conjunto de datos $\tilde{X}$ dado un objeto de la clase \verb|bpwpm| y los prueba contra las etiquetas reales $\tilde{y}$.
\end{itemize}

\subsubsection*{Funciones matemáticas}
Funciones auxiliares relacionadas con procedimientos matemáticos más complejos. Encontradas en el archivo \verb|math_utils.R|.
\begin{itemize}[label = {}]
	\item \verb|calculate_Phi|($X,M,J,K,d,\tau$): calcula la expansión de bases $\Psi_j(X,\P)$ para $j = 1,\ldots,d$ y cada una de las observaciones con base en los parámetros $\MJK$.
	\item \verb|calc_F|($\Psi,w,d$,\verb|intercept|): calcula la matriz $F = \Psi \wsn$ donde cada columna representa la transformación $f_j$.
	\item \verb|log_loss|($\ysn,\hat{\mathbf{p}},\ldots $): calcula la función \textit{log-loss} dados los valores reales $\ysn$ y las probabilidades ajustadas $\hat{p}$.
	\item \verb|mode|($x$): calcula la moda de un vector $x$.
	\item \verb|calc_proy|($F,\beta$): calcula el vector de medias para cada observación $f=F\beta$.
	\item \verb|model_proy|($\tilde{X}$, \verb|params|): calcula la función de proyección $f$ para un conjunto de datos $\tilde{X}$, pueden ser con los que se entrenaron los parámetros o un conjunto de datos nuevos.
	\item \verb|post_probs|($\tilde{X}$, \verb|params|): calcula la probabilidad posterior de la clasificación. 
	\item \verb|acurracy|($\tilde{y},\hat{\mathbf{p}}$): calcula la precisión total del modelo.
	\item \verb|contingency_table|($\tilde{y},\hat{\mathbf{p}}$): calcula la matriz de contingencia.
	\item \verb|ergodic_mean|(\verb|mcmc_chain|): calcula la media ergódica de una cadena MCMC.
\end{itemize}

\subsubsection*{Funciones útiles}
Funciones auxiliares para la simplificación de procesos en en las funciones de más alto nivel. Encontradas en el archivo \verb|utils.R|.
\begin{itemize}[label={}]
	\item \verb|thin_chain|(\verb|mcmc_chain, thin, burn_in|): dada una matriz de una cadena MCMC esta función auxiliar recorta y adelgaza la cadena.
	\item \verb|thin_bpwpm|(\verb|bpwpm, thin, burn_in|): adelgaza todos los parámetros de un modelo \verb|bpwpm| y regresa un objeto del mismo tipo. 
	\item \verb|post_params|(\verb|bpwpm, thin, burn_in, type|): dado un objeto de la clase \verb|bpwpm|, la función hace la estimación puntual de los parámetros del modelo $\beta$ y $\wsn$ utilizando el tipo de estimación puntual (media, moda o mediana) especificado en \verb|type|. Además, recorta y adelgaza la cadena y regresa un objeto conteniendo todos los parámetros con clase \verb|bpwpm_params|.
\end{itemize}

\subsubsection*{Funciones gráficas}
Funciones que habilitan el análisis gráfico de los datos y el modelo de forma rápida y sencilla. Existen 3 funciones importante que toman el papel \textit{envoltorio}\footnote{wrapper function} para las demás: \verb|plot.bpwpm, plot.bpwpm_predictions| y \verb|plot_2D|. Encontradas en el archivo \verb|plot_funcs.R|.

\begin{itemize}[label = {}]
	\item \verb|plot.bpwpm|(\verb|object, n|): para un objeto del tipo \verb|bpwpm|, grafica las trazas y los histogramas para los parámetros $\beta$ y cada $w_j$. 
	\item \verb|plot_chains|(\verb|mcmc_chains, n, title|): grafica la traza de una cadena MCMC.
	\item \verb|plot_hist|(\verb|mcmc_chains, n, title|): grafica el histograma de una cadena MCMC.  
	\item \verb|plot.bpwpm_predictions|(\verb|object, |$\ldots$): dado un objeto del tipo \verb|bpwpm_prediction|, grafica las $f_j$ del modelo ya estimado. Se usa como \textit{wraper} para la función \verb|plot_each_F|. 
	\item \verb|plot_each_F|($\ysn, X, F$): grafica cada una de las las funciones $f_j$. 
	\item \verb|plot_2D|($\ysn, X$ \verb|bpwpm_params|, $n$, \verb|alpha|): dados los datos en 2D, se realizan todas las gráficas posibles para este caso particular de los modelos. El parámetro \verb|alpha| controla la transparencia de los puntos proyectados y el parámetro $n$ la finura de la malla. 
	\item \verb|plot.2D_data|($\ysn, X$): grafica de puntos de los datos originales usando el paquete \verb|ggplot2|.
	\item \verb|plot.2D_proj|($\ysn, X$\verb|bpwpm.params, n, alpha|): grafica la proyección de $f$ en el espacio de covariables $\mathcal{X}$ para datos bivariados. Bueno para identificar las regiones de clasificación y visualizar los resultados del modelo.
	\item \verb|plot.3D_proj|($X$, \verb|bpwpm.params, n|): usando el paquete de \verb|lattice|, se crea una gráfica de malla para representar la función $f$ en 3D, únicamente se puede utilizar cuando se tengan datos en 2D. El parámetro $n$ controla la finura de la malla. 
	\item \verb|plot.ergodic_mean|(\verb|bpwpm, thin, burn_in|): grafica la media ergódica para las cadenas de un \verb|bpwpm| para todos los parámetros $\beta$ y $w_j$.
\end{itemize}

\subsubsection*{Funciones de resumen}
Estas tres funciones, son métodos S3 para la rápida exploración de los objetos del paquete. Se encuentran en el archivo \verb|summary_funcs.R|.

\begin{itemize}[label = {}]
	\item \verb|summary.bpwpm|(\verb|objeto y parámetros|): imprime en pantalla la información sobre la llamada del modelo y los principales resúmenes numéricos de las cadenas para los parámetros $\beta$ y $\wsn$.
	\item \verb|summary.bpwpm_params|(\verb|objeto y parámetros|): imprime la estimación sobre los parámetros posteriores $\hat{\beta}$ y $\hat{\wsn}$.
	\item \verb|summary.bpwpm_prediction|(\verb|objeto y parámetros|): resume e imprime la información sobre una predicción con el modelo. Esto es: la precisión, la medida \textit{log-loss}, el tipo de estimación puntual usada, la tabla de contingencia, los nodos y los parámetros posteriores entrenados. 
\end{itemize}

\end{document}