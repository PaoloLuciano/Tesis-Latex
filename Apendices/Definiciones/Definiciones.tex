\documentclass[../../Main/Main.tex]{subfiles}

\begin{document}
\subsection{Definiciones}
\begin{description}

	\item[Espacio con producto interno] Un espacio vectorial dotado de una estructura adicional llamada \textit{producto interno}: $\langle \cdot, \cdot \rangle$, que asocia cada par de vectores con una cantidad escalar sobre $F$. Es decir, $\langle \cdot, \cdot \rangle : V \times V \to F$. Que cumple, para $x,y,z$ vectores en $V$ y $a$ en $F$:	
	\begin{itemize}
		\item $\langle x,y\rangle =\overline{\langle y,x\rangle}$
		\item $\langle ax,y\rangle = a \langle x,y\rangle$
		\item $\langle x+y,z\rangle = \langle x,z\rangle + \langle y,z\rangle$
		\item $\langle x,x\rangle \geq 0 $ 
		\item $\langle x,x\rangle = 0 \Leftrightarrow x = \mathbf{0}$
	\end{itemize}

	\item[Espacio Funcional] Un espacio funcional es un espacio vectorial cuyos elementos son funciones.

	\item[Espacio Metrico] Un espacio métrico es un espacio donde la distancia (norma) inducida por el producto punto está definido sobre todos sus elementos. Norma: $||x|| = \sqrt{\langle x,x\rangle}$ la raiz no negativa del producto interno.
	
	\item[Espacio Métrico Completo] Un espacio métrico es completo si todas las secuencias de Cauchy, convergen a puntos dentro del espacio.  
	
	\item[Espacio Vectorial] Un espacio vectorial sobre un campo $F$ es un conjunto $V$, dotado de dos operaciones, \textit{suma} $+$ y \textit{multiplicación escalar} $\cdot$ que cumple los siguientes axiomas. Sean $x,y,z$ vectores en $V$, y $a,b$ escalares en $F$
	\begin{enumerate}
		\item $x+(y+z) = (x + y) + z$
		\item $x + y = y + x$
		\item $\exists 0 \in V$ tal que, $x + 0 = x$
		\item $\forall x \in V\quad \exists -x \in V$ tal que, $x + (-x) = 0$
		\item $a(bx) = (ab)x$ 
		\item $\exists 1 \in F$ tal que, $1x = x$
		\item $a(x+y) = ax + ay$
		\item $(a+b)x = ax + bx$
	\end{enumerate}	

	\item[Suavidad]:
	\item[Nodo]:
	\item[Continuidad casi en todas partes]:
	\item[Spline]:
	\item[Spline natural]:
	\item[Polinomio por partes]:
	\item[Regularización]:
	\item[Sobre ajuste]:
	\item[Identificabilidad]:
	\item[Expansión en bases truncadas]:
	\item[Polinomio base \textit{baseline}]:
	\item[Sampleo de Gibbs]:
	\item[Verosimilitud]:
	\item[Medida de probabilidad]: En este trabajo, se usa de forma equivalente a la función de densidad
	\item[Ergodicidad]: recurrente, aperiodico e irreducible
	\item[Cadena de Markov]:
	\item[Distribución límite]:
	\item[Distribución estacionaria]:
	\item[Distribución no informativa]:
	\item[Ortogonalidad] Dos elementos son ortogonales (en cierto espacio) si $\langle x,y\rangle = 0$. Denotado $x\perp y$

\end{description}

\subsection{Teoremas}
\begin{theorem}
	Ergodicidad: 
\end{theorem}

\begin{theorem}
	Ergodicidad: 
\end{theorem}

\subsection{Abreviaciones}
\begin{description}
	\item[GAM]: \textit{Generalized aditive model}, Modelo aditivo generalizado. 
	\item[GLM]: \textit{Generalized linear model}, Modelo lineal generalizado
	\item[MCMC]: \textit{Markov Chain Monte Carlo}.
	\item[ML]: \textit{Machine Learning}, Aprendizaje de máquina.
	\item[RSS]: \textit{Residual sum of squares}, Suma de residuales cuadrados.
\end{description}
\end{document}
