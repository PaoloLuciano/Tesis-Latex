\documentclass[../Main/Main.tex]{subfiles}

\begin{document}
%%%%%%%%%%%%%%%%%%%%%%%%%%%%%%%%%%%%%%%%%%%%%%%%%%%%%%%%%%%%%%%%%%%%%
%							PORTADA
%%%%%%%%%%%%%%%%%%%%%%%%%%%%%%%%%%%%%%%%%%%%%%%%%%%%%%%%%%%%%%%%%%%%%

\begin{titlepage}
\begin{center}

\underline{\textsc{\Large Instituto Tecnológico Autónomo de México}}\\[3em]

%Figura
\begin{figure}[h]
\begin{center}
\includegraphics[width=8cm,height=2.3cm]{LOGO_ITAM.jpg}
\end{center}
\end{figure}

\vspace{2em}

\textsc{\LARGE \textbf{\tituloMayus}}\\[2em]

\textsc{\large Tesis}\\[1em]

\textsc{que para obtener el título de}\\[1em]

\textsc{LICENCIADO EN MATEMÁTICAS APLICADAS}\\[1em]

\textsc{presenta}\\[1em]

\textsc{\Large GIANPAOLO LUCIANO RIVERA}\\[1em]

\textsc{\large Asesor: JUAN CARLOS MARTÍNEZ-OVANDO}

\end{center}

\vspace*{\fill}
\textsc{Ciudad de México \hspace*{\fill} 2019}

\end{titlepage}

%%%%%%%%%%%%%%%%%%%%%%%%%%%%%%%%%%%%%%%%%%%%%%%%%%%%%%%%%%%%%%%%%%%%%
%							DECLARACIÓN LEGAL
%%%%%%%%%%%%%%%%%%%%%%%%%%%%%%%%%%%%%%%%%%%%%%%%%%%%%%%%%%%%%%%%%%%%%
\thispagestyle{empty}

\begingroup
``Con fundamento en los artículos 21 y 27 de la Ley Federal del Derecho de Autor y como titular de los derechos moral y patrimonial de la obra titulada ``\textbf{\tituloMayus}'', otorgo de manera gratuita y permanente al Instituto Tecnológico Autónomo de México y a la Biblioteca Raúl Bailléres Jr., la autorización para que fijen la obra en cualquier medio, incluido el electrónico, y la divulguen entre sus usuarios, profesores, estudiantes o terceras personas, sin que pueda percibir por tal divulgación una contraprestación''.

\centering

\hspace{1em}

\textsc{Gianpaolo Luciano Rivera}

\vspace{5em}

\rule[1em]{20em}{0.5pt} % Línea para la fecha

\textsc{Fecha}
 
\vspace{5em}

\rule[1em]{20em}{0.5pt} % Línea para la firma

\textsc{Firma}

\endgroup
\vspace*{\fill}

%%%%%%%%%%%%%%%%%%%%%%%%%%%%%%%%%%%%%%%%%%%%%%%%%%%%%%%%%%%%%%%%%%%%%
%							ABSTRACT
%%%%%%%%%%%%%%%%%%%%%%%%%%%%%%%%%%%%%%%%%%%%%%%%%%%%%%%%%%%%%%%%%%%%%
\begin{abstract}
	\begin{singlespace}
	En respuesta al cambiante mundo del \textit{Aprendizaje de Máquina} se desarrolla desde sus cimientos un modelo aplicable a esta categoría. El modelo, no es más que un modelo lineal generalizado, particularmente un probit, que busca la predicción de variables binarias. A este, se le añadió un proyector aditivo que transforma de forma no lineal a las covariables para lograr detectar patrones más complejos. La transformación esta basada en polinomios por partes de continuidad arbitraria los cuales se estudian en detalle. Bajo el paradigma de aprendizaje bayesiano, se desarrolla un algoritmo eficiente para la estimación de los parámetros. Posteriormente, este algoritmo se implementa en un paquete para el software estadístico \verb|R|. Usando este paquete, finalmente, se prueba y se valida el modelo, presentando una variedad de resultados que exponen de forma clara e intuitiva las capacidades de el modelo. 
\vfill
	\noindent \textbf{Palabras clave}: modelos lineales generalizados, probit, modelos aditivos, bayesiano, no lineal, machine learning, splines, polinomios por partes
	\end{singlespace}
\end{abstract}
\clearpage

%%%%%%%%%%%%%%%%%%%%%%%%%%%%%%%%%%%%%%%%%%%%%%%%%%%%%%%%%%%%%%%%%%%%%
%							DEDICATORIA
%%%%%%%%%%%%%%%%%%%%%%%%%%%%%%%%%%%%%%%%%%%%%%%%%%%%%%%%%%%%%%%%%%%%%
\thispagestyle{empty}
\setcounter{page}{0} %Para que la dedicatoria no tengan número de hoja 

\begin{flushright}
	\null\vspace{\stretch{1}}
	\emph{a mi mamá Irma} 
	\vspace{\stretch{2}}\null
\end{flushright}
\clearpage

\end{document}