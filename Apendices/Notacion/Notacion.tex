\documentclass[../../Main/Main.tex]{subfiles}

\begin{document}

Se recomienda usar esta sección únicamente como referencia; a lo largo del texto principal se deriva y se presenta una descripción más detallada de cada uno de los símbolos usados. 

\textbf{Datos y variables}
\begin{itemize}[label={}]
	\item $y_i \in \{0,1\}\quad \forall i = 1\ldots,n$: variables de respuesta binarias. Usualmente representadas por el vector $\mathbf{y} = (y_1,\ldots,y_n)^t$
	\item $\xni \in \mathcal{X}^d \subseteq \mathbb{R}^d\quad \forall \; i = 1\ldots,n$: covariables o regresores. Si se usa por si sola $x$ o $\xsn$ (vector), ésta representa una variable arbitraria. Si se habla de toda la matriz de datos, se denota por $\xmat \in \mathcal{X}^{n\times d}\subseteq \mathbb{R}^{n\times d}$. Junto con las $y_i$, se tienen los datos para el modelo: $\{(y_i,\xni)\}_{i = 1}^n$
	\item $n \in \mathbb{N}$: número de observaciones en la muestra
	\item $d \in \mathbb{N}$: número de covariables, o dimensionalidad de estos
	\item $\lambda \in \mathbb{N}$: número total de términos en el modelo
	\item $\mathcal{X}^d \subseteq \mathbb{R}^d$: espacio de covariables. Formado por el producto punto de los rangos de cada variable: $\mathcal{X}^d = [a_1,b_1]\times[a_2,b_2]\times\ldots\times[a_d,b_d]$, donde $[a_j,b_j] \subset \mathbb{R}$ es un intervalo compacto en los reales
\end{itemize}

\textbf{Específicos del modelo}
\begin{itemize}[label={}]
	\item $z_i \sim \mathcal{N}(\cdot) \quad \forall i = 1,\ldots,n$: variables latentes del modelo cuya distribución es normal. En su forma vectorial: $\zsn = (z_1,\ldots,z_n)^t$ 
	\item $\eta(\xsn)$: función aditiva de predicción
	\item $f_j(x_j) \quad \forall j = 1,\ldots,d$: polinomios por partes
	\item $\betabf = (\beta_0, \beta_1,\ldots,\beta_\lambda)^t$: vector de parámetros por estimar. Si se le añade tilde entonces el contador comienza en uno y el vector no contiene el parámetro independiente, es decir: $\tilde{\betabf} = (\beta_1,\ldots,\beta_\lambda)^t$
	\item $\Psi_l(\cdot) \quad \forall l=1,\ldots,\N$: funciones bases para la expansión en polinomios por partes de $f_j$. Ver \eqref{ec:ExpansionBases} y (\ref{ec:ExpBase_NEstrella}). En ocasiones, todas las funciones base se organizan en una matriz $\widetilde{\Phi}$
	\item $\N$: número total de funciones base. Ver ecuación (\ref{ec:PoliFinal}) para su expansión final. Si se usa $N$ sin el asterisco denota de igual forma un número de funciones base arbitrario.
	\item $M$: tamaño de la base para los polinomios por partes, por lo tanto, M-1 indica el grado de los polinomios
	\item $J$: número de sub-intervalos en los que se parte cada intervalo $[a_j,b_j]$
	\item $K$: número de restricciones de continuidad impuestas a los polinomios por partes
	\item $\mathcal{P}_j = \left\{\t_1,\ldots,\t_{J-1}\right\} \quad \forall j = 1,\ldots,d$: partición del espacio de la dimensión $j$
	\item $\t$: nodos, se tienen un total de $d\times (J-1)$ nodos acomodados en una matriz de igual tamaño.
\end{itemize}

\textbf{Contadores e índices}
\begin{itemize}[label={}]
	\item $i = 1,\ldots,n$: contador usado para denotar un conjunto de observaciones
	\item $j = 1,\ldots,d$: contador usado para denotar el conjunto de covariables. Usualmente se hace referencia a la dimensión arbitraria $j$
	\item $k$: contador usado para denotar el número de iteración en el algoritmo, i.e. $k = 0,1,2,3,\ldots$
	\item $l = 1,\ldots,\N$: contador asociado al número de funciones base total en la expansión de polinomios por partes $\N$
	\item $\hati = 1,\ldots,M-1$: contador asociado al número de funciones base para cada subintervalo, $M$, en las expansiones de polinomios truncados
	\item $\hatj = 1,\ldots,J-1$ contador asociado al número de funciones base para cada subintervalo (parámetro $M$) en las expansiones de polinomios truncados
\end{itemize}
	
\textbf{Probabilidad}
\begin{itemize}[label={}]
	\item $F(\cdot)$: Función de distribución arbitraria de la familia exponencial
	\item $\mathcal{N}(\,\cdot\,|\mu,\sigma^2)$: distribución normal con su correspondiente parametrización de media y varianza. Se utiliza la misma notación para su forma vectorial añadiendo un subíndice para denotar su dimensionalidad: $\mathcal{N}_{\cdot}(\cdot|\bm{\mu},\Sigma)$ con su correspondiente vector de medias $\bf{\mu}$ y vector de varianza covarianza $\Sigma$
	\item $\Phi(\cdot):\mathbb{R}\rightarrow(0,1)$: la función de distribución acumulada de una distribución normal estándar $\mathcal{N}(\,\cdot\,| 1,0)$, con su correspondiente función inversa $\Phi^{-1}$
	\item $\Be(\,\cdot\,|p)$: distribución bernoulli con probabilidad de éxito $p$
	\item $p \in [0,1]$: probabilidad arbitraria
	\item $g(\cdot)$: función liga, ver diagrama \ref{fig:DiagramaFuncLiga}
	\item $\epsilon$: errores aleatorios, usualmente distribuidos $\epsilon\sim\mathcal{N}(\epsilon\,|\,\mu,\sigma^2)$
	\item $P(\cdot),\, \E[\cdot],\, \Var[\cdot]$: medida de probabilidad, operadores de esperanza y varianza respectivamente
	\item $\theta \in \Theta$ parámetros canónicos de distribuciones exponenciales, con $\Theta$ su correspondiente espacio
	\item $\pi(\cdot)$: función de densidad
	\item $\propto$: operador de proporcionalidad
%	\item $S(\cdot|\cdot)$: función de suavizamiento
	\item $\rho\,$: correlación lineal de Pearson
	\item $\L$: función de pérdida
\end{itemize}
	
\textbf{Algoritmo}
\begin{itemize}[label={}]
	\item $\nsim$: número de simulaciones realizadas en el algoritmo
	\item $k^*$:  periodo de \textit{burn-in}; número de observaciones por descartar
	\item $\kthin$: parámetro de adelgazamiento 
%	\item $r_{j^*}$: residuales parciales para alguna $j^*$ en particular. 
\end{itemize}

\textbf{Misceláneos}
\begin{itemize}[label={}]
	\item $h(\cdot)$: función arbitraria
	\item $h^{(k)}$: (k)-ésima derivada de la función $h$.
	\item $s: \mathbb{R} \rightarrow (0,1)$: familia de funciones sigmoidales
	\item $I$: función indicadora
	\item $(\cdot)_{+}$: función parte positiva
%	\item $\mathbf{1}$: vector de números uno
	\item $ll$: función \textit{log-loss} (Ver pág. \pageref{ec:LogLoss})
	\item El símbolo $\hat{\cdot}$ se usa para indicar que se trata de una variable estimada, i.e. $\hat{y}$ es la estimación de las variables correspondientes $y$
\end{itemize}

\subsection*{Abreviaciones}
\begin{description}
	\item[ANOVA]:\textit{ANalysis Of VAriance}, modelos de análisis de varianza
	\item[GAM]: \textit{Generalized aditive model}, modelo aditivo generalizado
	\item[GLM]: \textit{Generalized linear model}, modelo lineal generalizado
	\item[MCMC]: \textit{Markov Chain Monte Carlo}, cadena de Markov Montecarlo
	\item[ML]: \textit{Machine Learning}, aprendizaje de máquina
		\item[OLS]: \textit{Ordinary Least Squares}, método de ajuste de mínimos cuadrados ordinarios, el cual utiliza a la función RSS como función objetivo
	\item[RSS]: \textit{Residual sum of squares}, suma de residuales cuadrados
\end{description}

\end{document}

