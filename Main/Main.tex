%%%%%%%%%%%%%%%%%%%%%%%%%%%%%%%%%%%%%%%%%%%%%%%%%%%%%%%%%%%%%%%%%%%%%
%							PREAMBLE
%%%%%%%%%%%%%%%%%%%%%%%%%%%%%%%%%%%%%%%%%%%%%%%%%%%%%%%%%%%%%%%%%%%%%
\documentclass[pdftex,11pt]{report}

% Tamaño Página normal
\usepackage[left=2cm,top=1.5cm,right=2cm,bottom=2cm]{geometry} 
% Para ya formato bueno Tesis ITAM
%\usepackage[paperwidth = 18cm, paperheight = 22.5cm, bottom = 2.5cm, right = 2cm, left = 2cm]{geometry} 
\usepackage[spanish,es-tabla]{babel} % Español y la palabra tabla para las tablas...
\decimalpoint % Malditos anormales que usan comas
\usepackage[utf8]{inputenc} % Acentos y demás
\usepackage{amsmath} % Simbolos y cosas bonitas
\usepackage{amsfonts} 
\usepackage{array} % Tablas bonitas
\usepackage{arydshln} % Lineas para tablas aún más bonitas
\usepackage{multirow,bigdelim} % Ya la neta por forzado
\usepackage{listings} % Insertar 3		Código
\usepackage{setspace} % Para los espacios de la por tada
\renewcommand{\baselinestretch}{1.5} % Espaciado de lineas
%\linespread{2} 

% Imagenes
\usepackage[pdftex]{graphicx} 
\usepackage{subcaption}
% El primer path es relativo al Main y el otro es relativo a los subfiles
\graphicspath{{../Imagenes/}{../Imagenes/}} 

\usepackage{hyperref} %Hipervinculos
\usepackage{pdflscape} %Poner algunas páginas en Horizontal \begin{landscape}
\usepackage{multicol}  % Latex a Dos Columnas
\usepackage{float} % Poner mejor las imágenes.
\usepackage[makeroom]{cancel} % Strikeout diagonal

% Cuando quería hacer un glosario de terminos y definiciones
%\usepackage{glossaries}

% Estructura para el main file
\usepackage{subfiles}
 
% Para las pequeñas gráficas
\usepackage{tikz}
\usetikzlibrary{shapes, arrows, babel}

% Usar cuando sea relevante 
\setlength\parindent{0pt} % Quitar las sangías
% \setlength{\intextsep}{1pt plus 1pt minus 1pt} % Pámetro de texto entre figura e imágen.

% Cambiamos el espacio entre los footnotes y el texto
\setlength{\skip\footins}{25pt plus 5pt minus 5pt}

% Mejoro el espacioe entre los floats y el texto
\setlength{\textfloatsep}{25pt plus 5pt minus 5pt}

% Biblografía (uso Biblatex-Chicago con backend en biber) 	
\usepackage[authordate, backend = biber]{biblatex-chicago}
\addbibresource{Tesis.bib} 

 \usepackage[nottoc,numbib]{tocbibind}

%%%%%%%%%%%%%%%%%%%%%%%%%%%%%%%%%%%%%%%%%%%%%%%%%%%%%%%%%%%%%%%%%%%%%
%							COMMANDS
%%%%%%%%%%%%%%%%%%%%%%%%%%%%%%%%%%%%%%%%%%%%%%%%%%%%%%%%%%%%%%%%%%%%%

\newcommand{\tituloMayus}{UN MODELO PROBIT BAYESIANO NO LINEAL}

\newcommand{\xni}{\mathbf{x}_i} % Vector de datos X's
\newcommand{\xsn}{\mathbf{x}}
\newcommand{\xmat}{\mathbf{X}}
\newcommand{\ysn}{\mathbf{y}}
\newcommand{\zsn}{\mathbf{z}}
\newcommand{\wsn}{\mathbf{w}}
\newcommand{\fsn}{\mathbf{f}}
%\newcommand{\betasn}{\boldsymbol{\beta}}
\renewcommand{\t}{\tau}

\newcommand{\tsn}{\boldsymbol{\tau}}
\newcommand{\E}{\mathbb{E}}
\newcommand{\Var}{\mathbb{V}}
\newcommand{\Be}{\text{Be}}
\newcommand{\N}{N^*}
\newcommand{\MJK}{M,\;J\text{ y }K}
\newcommand{\nsim}{N_{\text{sim}}}
\newcommand{\kthin}{k_{\text{thin}}}
\newcommand{\iter}[1]{^{(#1)}} % Superscript

\newcommand{\intinfty}{\int_{-\infty}^\infty}
\renewcommand{\P}{\mathcal{P}}
\renewcommand{\H}{\mathcal{H}}

% Definiendo el enviroment Teorema
\newtheorem{theorem}{Teorema}

%%%%%%%%%%%%%%%%%%%%%%%%%%%%%%%%%%%%%%%%%%%%%%%%%%%%%%%%%%%%%%%%%%%%%
%							DOCUMENTO
%%%%%%%%%%%%%%%%%%%%%%%%%%%%%%%%%%%%%%%%%%%%%%%%%%%%%%%%%%%%%%%%%%%%%

\begin{document} 
%\subfile{../PortadaYMas/Portada}

%%%%%%%%%%%%%%%%%%%%%%%%%%%%%%%%%%%%%%%%%%%%%%%%%%%%%%%%%%%%%%%%%%%%%
%							ÍNDICE
%%%%%%%%%%%%%%%%%%%%%%%%%%%%%%%%%%%%%%%%%%%%%%%%%%%%%%%%%%%%%%%%%%%%%
\pagebreak
\tableofcontents

%%%%%%%%%%%%%%%%%%%%%%%%%%%%%%%%%%%%%%%%%%%%%%%%%%%%%%%%%%%%%%%%%%%%%
%							CAPITULOS
%%%%%%%%%%%%%%%%%%%%%%%%%%%%%%%%%%%%%%%%%%%%%%%%%%%%%%%%%%%%%%%%%%%%%

\chapter{Introducción} \label{cap:Intro}
\subfile{../Introduccion/Introduccion}

\chapter{Modelo en su forma matemática} \label{cap:Modelo}
\subfile{../Modelo/Modelo}

\chapter{Paradigma bayesiano e implementación} \label{cap:BayesAlgoritmo}
\subfile{../BayesAlgoritmo/BayesAlgoritmo}

\chapter{Ejemplos y resultados} \label{cap:EjYRes}
%HC SVNT DRACONES
\subfile{../EjemYRes/EjemYRes}

\chapter{Conclusiones} \label{cap:Conclusiones}
\subfile{../Conclusiones/Conclusiones}

%%%%%%%%%%%%%%%%%%%%%%%%%%%%%%%%%%%%%%%%%%%%%%%%%%%%%%%%%%%%%%%%%%%%%
%							APÉNDICES Y BIBILOGRAFÍA
%%%%%%%%%%%%%%%%%%%%%%%%%%%%%%%%%%%%%%%%%%%%%%%%%%%%%%%%%%%%%%%%%%%%%

\appendix
\chapter{Análisis Funcional} \label{ap:AnalisisFunc}
\subfile{../Apendices/AnalisisFunc/AnalisisFunc}

\chapter{Paquete en R. Desarrollo y Lista de Funciones} \label{ap:Paquete}
\subfile{../Apendices/Paquete/Paquete}

\chapter{Notación} \label{ap:Notacion}
\subfile{../Apendices/Notacion/Notacion}

\chapter{Definiciones} \label{ap:Definiciones}
\subfile{../Apendices/Definiciones/Definiciones}

%Bibliografía
\addcontentsline{toc}{chapter}{Bibliografía}
\printbibliography

\end{document}