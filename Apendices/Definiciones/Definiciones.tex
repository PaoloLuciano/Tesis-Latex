\documentclass[../../Main/Main.tex]{subfiles}

\begin{document}
\section*{Definiciones}
\begin{description}

	\item[Suavidad]:
	\item[Nodo]:
	\item[Continuidad casi en todas partes]:
	\item[Spline]:
	\item[Spline natural]:
	\item[Polinomio por partes]:
	\item[Regularización]:
	\item[Sobre ajuste]:
	\item[Identificabilidad]:
	\item[Expansión en bases truncadas]:
	\item[Polinomio base \textit{baseline}]:
	\item[Sampleo de Gibbs]:
	\item[Verosimilitud]:
	\item[Medida de probabilidad]: En este trabajo, se usa de forma equivalente a la función de densidad
	\item[Ergodicidad]: recurrente, aperiodico e irreducible
	\item[Cadena de Markov]:
	\item[Distribución límite]:
	\item[Distribución estacionaria]:
	\item[Distribución no informativa]:
	\item[Ortogonalidad] Dos elementos son ortogonales (en cierto espacio) si $\langle x,y\rangle = 0$. Denotado $x\perp y$

\end{description}

\section*{Teoremas}
\begin{theorem}
	Ergodicidad: 
\end{theorem}

\begin{theorem}
	Ergodicidad: 
\end{theorem}

\end{document}
